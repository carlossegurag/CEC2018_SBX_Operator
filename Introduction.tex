Evolutionary Algorithms (EA) have become a promising alternative in several real-world problems where the deterministic approches are not suitable.
%
Multi-objective Optimization Problems (MOPs) involves the simultaneous optimization of two or more objective functions that are usually in conflict.
%
A continuous minimization multi-objective problem can be defined as follows:
%
\begin{equation}\label{Base}
\begin{split}
&minimize \quad F(x) = (f_1(x), f_2(x), ..., f_m(x)) \\
&subject \quad to \quad x \in \Omega
\end{split}
\end{equation}
where $x = (x_1, ..., x_n) \in R^n$ indicate a decision variable vector, $n$ correspond to the number of decision variables, $\Omega$ is the feasible space, $F: \Omega \rightarrow R^m$ which consist of $m$ objective functions and $R^m$ is known as the \textit{objective space}.
%

%Particularly, a MOP which consists in minimization of the $m$ objective functions and given two decision variable vectors $x, y \in \Omega$, $x$ dominates $y$ denoted by $x \prec y$, iff $f_i(x)  \leq f_i(y)$ for all objectives $\{1,...,m\}$, and the objective vectors are different $F(x) \neq F(y)$.
%%
%Accordingly this, the solution $x$ is not worse than $y$ in any of the objectives and $x$ is strictly better than $y$ in at least one objective.
%%
%The Pareto dominance is defined as the set of the best solutions that are not dominated by any feasible solution.
%%
%A decision variable vector $x^* \in \Omega$ is known as the Pareto optimal solution if does not exist any solution $x \in \Omega$ that dominates $x^*$.
%%
%The Pareto set correspond to the set of all Pareto optimal decision vectors and the Pareto front are the images of the Pareto set.
%%
%Principally, the goal in multi-objective optimization is to obtain an approximation of the Pareto front.
%%
%Therefore is required to obtain diverse and converged solutions among the Pareto front.
%
In a minimization MOP with $m$ objective functions, and given two solutions $x$, $y \in \Omega$, $x$ dominates $y$, denoted by  $x \prec y$, if $f_i(x) \leq f_i(y)$ for all objectives $\{1, ..., m \}$, and $F(x) \neq F(y)$. 
%
This means that the solution $x$ is not worse than $y$ in any of the objectives and $x$ is strictly better than $y$ in at least one objective.
%
The Pareto dominance definition states that the best solutions of a multi-objective optimization problem are those whose objective vectors are not dominated by any other feasible vector. 
%
A solution $x^* \in \Omega$ is known as Pareto optimal solution if no other solution $x \in \Omega$ dominates $x^*$. 
%
The Pareto set is the set of all the Pareto optimal solutions and the Pareto front are the images of the Pareto set.
%
The goal of multi-objective optimization approaches is to obtain a proper approximation of the Pareto front. 
%
Particularly, a set of solutions that are diverse and close to the Pareto front are desired. 


In the last decade several categories of Multi-Objective Evolutionary Algorithms (MOEAs) have been arising \cite{Joel:Kalyanmoy,Joel:Coello}, among them, the Non-Dominated Sorting Genetic Algortihm II (NSGA-II) \cite{Joel:NSGAII}, the MOEA based on decomposition (MOEA/D) \cite{Joel:MOEAD}, the $S$-metric Selection Evolutionary Multi-objective Optimization Algorithm (SMS-EMOA) \cite{Joel:SMSEMOA}, these representative methods are considered as the state-of-the-art.

Differential Evolution (DE) is a popular and efficient Evolutionary Algorithm (EA) which has yielded better results than Genetic Algorithms (GAs) \cite{tuvsar2007differential, mi2010improved}.
%
However, in the literature is well known that several componentes of DE induces an aggresive convergence leading to several drawbacks in different scenarios such as in long-term exetucions.
%
Usually, the aggressive behaviour of DE provokes that the algorithm stagnates in some sub-optimal regions at the first stages of the execution.
%


To deal with these drawbacks, the most recent algorithms are designed with strategies where is considered the criteria stop, such as adaptive parameters \cite{brest2017single}, Linear Population Size Reduction (LPSR) \cite{brest2008population} and several mutation strategies \cite{das2011differential}.
%
Principally, these adaptive strategies are oriented to deal with the substantial disadvantage of setting the control parameters.
%
These are the crossover probability ($CR$) and the mutation factor scale ($F$).
%
%
Also, is important to take into account the implications between single-objective and multi-objective problems which are related with the decision variable space diversity \cite{kukkonen2007performance}.

%
On the other hand, GAs imply a less aggressive and flexible behavior provoking lowest quality solutions than DE in short-term executions both in continuous and discrete domains  \cite{tuvsar2007differential, mi2010improved}.
%
Nevertheless, the GAs have interesting behaviours when are considered in long-term executions.
%
Principally, because several componentes are based in a probability density distribution which provides an extensive search process.
%

In this paper, a novel recombination operator, is designed for tackling continuous multi-objective optimization problems (MOPs), which works effectively to enhance the search capability, guiding to a gradual change between exploration to intensification.
%
Therefore, several components of the Simulated Binary Crossover (SBX) are analyzed and modified, which aims to offer a suitable performance in long-term executions.
%

The rest of this paper is organized as follows.
%
A brief description of the state-of-the-art MOEAs and detailed revision of the SBX operator is carried out in section \ref{LiteratureReview}.
%In section \ref{LiteratureReview} is revised the literature related with the SBX operator, and a brief description of the state-of-the-art of MOEAs.
%
Some empirical studies of the SBX operator are realized.
%
Based in this stuides is proposed a dynamic variant of the SBX in section \ref{Proposal}.
%
The experimental validation of the proposal and some DE variants are showed in Section \ref{Experimental_Validation}.
%
Finally, conclusions and some lines of future work are given in Section \ref{Conclusions}.
